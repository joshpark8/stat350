\chapter{Continuous Probability Distributions}  % 6

\section{Probability Distribution for Continuous Random Variables - General}  % 6.1
\subsection*{Objectives}
\begin{enumerate}
    \item Describe the basis of the probability density function (pdf).
    \item Use the probability density function (pdf) and cumulative distribution function (cdf) of a continuous random variable to calculate probabilities and percentiles (median) of events.
    \item Be able to use a pdf to find the mean of a continuous random variable.
    \item Be able to use a pdf to find the variance of a continuous random variable.
\end{enumerate}

\subsection{Density curves and probabilities (pdf)}  % 6.1.1
Define the pdf $f(x)$ such that $\Dint f(x)\mathrm{d}x=1$. \\
Then, the probability that $a < X< b$ is
\begin{equation}P(a<X<b)=\dint{a}{b}f(x)\mathrm{d}x\end{equation}
\nt{When $X=a$, $\int_a^a f(x)$d$x=0$, so $P(X\leq a)=P(X<a)$}

\subsection{Properties}  % 6.1.2
A valid density curve must have the two following properties:
\begin{enumerate}
    \item $f(x)\geq 0$
    \item $\int_{-\infty}^\infty f(x)$d$x=1$
\end{enumerate}
\nt{Notice that $f(x)\leq 1$ need not be true; consider the function $g(x)=4$ on the interval [0, 0.25].}\\
\nt{In the case of $g(x)$, the bounds on the integral for property 2 must be adjusted}

\subsection{Mean and Variance}  % 6.1.3
\nt{The rules for the means and variances are the same for both discrete and continouous random variables; the only difference is how the values are computed.}
\begin{align}
     & \text{Discrete:}   & E(X) & =\mu_X=\sum xp(x)                             & E(g(X)) & =\sum g(x)p(x)                            \\
     & \text{Continuous:} & E(X) & =\mu_X=\int_{-\infty}^{\infty} xp(x)\text{d}x & E(g(X)) & =\int_{-\infty}^{\infty}g(x)p(x)\text{d}x
\end{align}
Recall the formula for variance of discrete random variables
\begin{align}
    \begin{split}
        Var(X) & =E[{(X-\mu_X)}^2]=\sum{(x_i-\mu_X)}^2\cdot p_i                  \\
               & =E[{(X-\mu_X)}^2]=\int_{-\infty}^\infty{(x_i-\mu_X)}^2\cdot p_i
    \end{split} \\
    Var(X)   & =E(X^2)-(E(X))^2                                              \\
    \sigma_X & =\sqrt{Var(X)}
\end{align}
Equation 6.4 is recommended as it is computationally much easier to evaluate.

\subsection{Cumulative Distribution Function (cdf)}  % 6.1.4
The cumulative distribution function (cdf) is the probability that the random variable will be less than or equal to some value. It is written $F(X)$ and the formula is
\begin{equation}
    F(x)=P(X\leq x)=\int_{-\infty}^{s} f(x)ds
\end{equation}
\nt{The variable of integration is changed to be some dummy variable $s$, as the bounds of a definite integral can not be a function of the variable of integration.}\\
To recap, we now have
\begin{align}
    p(x) & = \text{probability mass function}        \\
    f(x) & = \text{probability density function}     \\
    F(x) & = \text{cumulative distribution function}
\end{align}

\subsection{Percentiles}  % 6.1.5
For continuous distributions, percentiles are much simpler to compute. Given that $0<p<1$, the $100p$th percentile for a value $x$ can be computed with the integral
\begin{equation}
    p=\int_{-\infty}^{x} f(x)ds
\end{equation}
Note that this integral is the same as the cdf. Thus if the cdf is already known, we can simply find when $F(x)=p$. Again, the $100p^{\text{th}}$ percentile is when $100p$ percent of the data is less than $p$, and the rest is above\\
\nt{The median occurs when $p=0.5$. Hence,}
\begin{equation}
    p=0.5=\int_{-\infty}^{\mu'} f(x)\mathrm{d}x=F(\mu')\text{ where } \mu'=\tilde{\mu}
\end{equation}

\section{Normal Distribution}  % 6.2
\begin{equation}
    f(x)=\frac{1}{\sigma \sqrt{2\pi}}e^{-\frac{(x-\mu)^2}{2\sigma^2}} \mathrm{ where } -\infty<\mu<\infty
\end{equation}
don't need to know closed form \\
defined as long as we know the variance $\sigma^2$ and the mean $\mu$ \\
thus $X\sim N(\mu, \sigma^2)$
\subsection{Normal Distribution}  % 6.2.1
shape: symmetric; bell shaped; unimodal \\
variance $\sigma^2$ is spread; mean is also median since symmetric \\
curve is concave down when $x$ is $\mu\pm\sigma$ otherwise concave down

\subsection{Standardization}  % 6.2.2
empirical rule - 68\% of data within 1 stdev; 95 within 2; 99.7 within 3 \\
standard normal distribution: $\mu=0; \sigma=1$\\
\dfn{$z=f(z)$}\\
probability of certain outcome $x$ $=$ cdf of $z$; values are in z table \\

\subsection{Using the z-table}  % 6.2.3
\begin{equation}
    P(-2.77<Z<1.54)=P(Z<1.54)-P(Z<-2.77)=0.9382 - 0.0028 = 0.9354
\end{equation}

\subsection{Probabilities}  % 6.2.4
normalize any given distribution to be standard normal with
\begin{equation}
    z=\frac{x-\mu}{\sigma}
\end{equation}
then $z$ is z-score; tells how many standard deviations $x$ is from the mean $\mu$
\subsection{Percentiles}  % 6.2.5
\ex{89th percentile = $P(Z<b)=0.89$} \\
must find corresponding value in z-table (don't interpolate; use printed value) \\
now un-normalize z score by
\begin{equation}
    z=\frac{x-\mu}{\sigma}  \implies x=\mu+\sigma z
\end{equation}

\subsubsection{Symmetry}
How do we find $P(\mu-b\leq X\leq \mu+b)$ (symmetry around the mean)?\\
Let $P(\mu-b\leq X<\leq \mu+b)=C$. Then each little ``wedge'' on each end is \(\frac{1-c}{2}\)\\
Then upper bound of $C$ is \(P(X<\mu+b)=1-\frac{1-C}{2}\); and lower bound is $\frac{1-C}{2}$

\section{Determining Normality}  % 6.3


\subsection{Normal probability plots}  % 6.3.1
Ways to check normality
\begin{enumerate}
    \item graph and compare
    \item backward empirical rule to see if proportions match
    \item ratio of IQR:s shoudl be about 1.4
    \item normal probability plot (QQ) (best method)
    \item inference
\end{enumerate}

\subsection{Conceptual Procedure for Normal Probability Plots}  % 6.3.2
idea: if data fits distribution, the percentiles should correspond with the percentiles of that distribution
\begin{enumerate}
    \item arrange from smallest to largest
    \item record corresponding percentiles (complicated)
    \item find z value
    \item plot original data point vs new z point
\end{enumerate}
If data is normal, we can use \(x\) and \(z\) to obtain the mean and standard

\section{Uniform Distribution}  % 6.4
continuous distribution; all points are distributed evenly between $a, b$. \\
density curve:
\begin{equation}
    f(x) = \begin{cases}
           \frac{1}{b-a} & a<x<b\\
           0 & else
        \end{cases}
\end{equation}
Mean is\begin{equation}
    \EE(X)=\frac{a+b}{2}
\end{equation}
standard deviation \begin{equation}
    \sigma_X=\frac{b-a}{\sqrt{12}}
\end{equation}
\ex{200 boxes an hour; distribution is uniform from 18.2 to 20.4 (nearest tenth)}
\begin{enumerate}
    \item what is the probability that the package weighs less than 20 lbs? \begin{equation}
        P(X<20)=\int_{18.2}^{20}\frac{1}{20.4-18.2}dx=\int_{18.2}^{20}\frac{1}{2.2}dx=0.818
    \end{equation}
    \item what are the mean and standard deviation of the weights?\begin{equation}
        \EE(X)=\frac{18.2+20.4}{2}=19.3
    \end{equation}\begin{equation}
        \sigma_X=\frac{20.4-18.2}{\sqrt{12}}=0.635
    \end{equation}
\end{enumerate}

\section{Exponential Distribution}  % 6.5
exponential distribution pdf:\begin{equation}
    f(x)=\begin{cases}
        \lambda e^{-\lambda x} & x \geq 0\\
        0 & else
    \end{cases}
\end{equation}
cdf:\begin{equation}
    F(x)=\begin{cases}
        0 & x<0 \\
        1-e^{\lambda x} & x\geq 0
    \end{cases}
\end{equation}
\begin{equation}
    E(X)=\frac{1}{\lambda} \qquad\qquad Var(X) = \frac{1}{\lambda\sq} \qquad\qquad \sigma_X =\frac{1}{\lambda}
\end{equation}
% \section{Other continuous distributions (Optional)}  % 6.6
% \subsection{Gamma Distribution}  % 6.6.1
% \subsection{Beta Distribution}  % 6.6.2
% \subsection{Weibull Distribution}  % 6.6.3
% \subsection{Lognormal Distribution}  % 6.6.4
