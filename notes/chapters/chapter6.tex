\chapter{Continuous Probability Distributions}  % 6

\section{Probability Distribution for Continuous Random Variables - General}  % 6.1
\subsection*{Objectives}
\begin{enumerate}
    \item Describe the basis of the probability density function (pdf).
    \item Use the probability density function (pdf) and cumulative distribution function (cdf) of a continuous random variable to calculate probabilities and percentiles (median) of events.
    \item Be able to use a pdf to find the mean of a continuous random variable.
    \item Be able to use a pdf to find the variance of a continuous random variable.
\end{enumerate}

\subsection{Density curves and probabilities (pdf)}  % 6.1.1
Define the pdf $f(x)$ such that $\int_\infty^\infty f(x)\mathrm{d}x=1$. \\
Then, the probability that $a < X < b$ is
\[P(a<X<b)=\int_a^bf(x)\mathrm{d}x\]
\nt{When $X=a$, $\int_a^a f(x)$d$x=0$, so $P(X\leq a)=P(X<a)$}

\subsection{Properties}  % 6.1.2
A valid density curve must have the two following properties:
\begin{enumerate}
    \item $f(x)\geq 0$
    \item $\int_{-\infty}^\infty f(x)$d$x=1$
\end{enumerate}
\nt{Notice that $f(x)\leq 1$ need not be true; consider the function $g(x)=4$ on the interval [0, 0.25].}\\
\nt{In the case of $g(x)$, the bounds on the integral for property 2 must be adjusted}

\subsection{Mean and Variance}  % 6.1.3
\nt{The rules for the means and variances are the same for both discrete and continouous random variables; the only difference is how the values are computed.}
\begin{align}
     & \text{Discrete:}   & E(X) & =\mu_X=\sum xp(x)                             & E(g(X)) & =\sum g(x)p(x)                            \\
     & \text{Continuous:} & E(X) & =\mu_X=\int_{-\infty}^{\infty} xp(x)\text{d}x & E(g(X)) & =\int_{-\infty}^{\infty}g(x)p(x)\text{d}x
\end{align}
Recall the formula for variance of discrete random variables
\begin{align}
    \begin{split}
        Var(X) & =E[{(X-\mu_X)}^2]=\sum{(x_i-\mu_X)}^2\cdot p_i                  \\
               & =E[{(X-\mu_X)}^2]=\int_{-\infty}^\infty{(x_i-\mu_X)}^2\cdot p_i
    \end{split} \\
    Var(X)   & =E(X^2)-(E(X))^2                                              \\
    \sigma_X & =\sqrt{Var(X)}
\end{align}
Equation 6.4 is recommended as it is computationally much easier to evaluate.

\subsection{Cumulative Distribution Function (cdf)}  % 6.1.4
The cumulative distribution function (cdf) is the probability that the random variable will be less than or equal to some value. It is written $F(X)$ and the formula is
\begin{equation}
    F(x)=P(X\leq x)=\int_{-\infty}^{x} f(x)ds
\end{equation}
\nt{The variable of integration is changed to be some dummy variable $s$, as the bounds of a definite integral can not be a function of the variable of integration.}\\
To recap, we now have
\begin{align}
    p(x) & = \text{probability mass function}        \\
    f(x) & = \text{probability density function}     \\
    F(x) & = \text{cumulative distribution function}
\end{align}

\subsection{Percentiles}  % 6.1.5
For continuous distributions, percentiles are much simpler to compute. Given that $0<p<1$, the $100p$th percentile for a value $x$ can be computed with the integral
\begin{equation}
    p=\int_{-\infty}^{x} f(x)ds
\end{equation}
Note that this integral is the same as the cdf. Thus if the cdf is already known, we can simply find when $F(x)=p$. Again, the $100p^{\text{th}}$ percentile is when $100p$ percent of the data is less than $p$, and the rest is above\\
\nt{The median occurs when $p=0.5$. Hence,}
\begin{equation}
    p=0.5=\int_{-\infty}^{\mu'} f(x)\mathrm{d}x=F(\mu')\text{ where } \mu'=\tilde{\mu}
\end{equation}

\section{Normal Distribution}  % 6.2
\subsection{Distribution}  % 6.2.1
\subsection{Standardization}  % 6.2.2
\subsection{Using the z-table}  % 6.2.3
\subsection{Probabilities}  % 6.2.4
\subsection{Percentiles}  % 6.2.5

\section{Determining if a distribution is normal}  % 6.3
\subsection{Normal probability plots}  % 6.3.1

\section{Uniform Distribution}  % 6.4
\subsection{Distribution}  % 6.4.1

\section{Exponential Distribution}  % 6.5
\subsection{Distribution}  % 6.5.1

\section{Other continuous distributions (Optional)}  % 6.6
\subsection{Gamma Distribution}  % 6.6.1
\subsection{Beta Distribution}  % 6.6.2
\subsection{Weibull Distribution}  % 6.6.3
\subsection{Lognormal Distribution}  % 6.6.4
