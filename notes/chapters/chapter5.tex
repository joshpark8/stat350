\chapter{Random Variables and Discrete Probability Distributions}  % 5

    \section{Random Variables}  % 5.1
        \subsection{Random Variables}  % 5.1.1
            \dfn{A \emph{random variable} is a numerical characteristic obtained from a random experiment. So, random variables are functions and follow all properties of mathematical functions.}

        \subsection{Probability Distributions - pmf}  % 5.1.2
            \dfn{The probability distributions of a random variable is called the \emph{probability mass function (pmf)}. In symbols, $p(x)=P(X=x)$}

        \subsection{Properties}  % 5.1.3
            Pmfs are valid probability distributions, so they follow the axioms of probability.
            \begin{enumerate}
                \item $0\leq p_i\leq 1$. Each probability lies between 0 and 1.
                \item $\sum_i p_i(x)=1$. The sum of all probabilities is 1.
            \end{enumerate}

    \section{Expected Value and Variance}  % 5.2

        \subsection{Expected Value}  % 5.2.1
            \dfn{The \emph{expected value} of a discrete random variable $X$ is the weighted average of each value.}
            \begin{equation} E(X)=\mu_X=\sum^m_{i=1} x_ip_i \end{equation}

        \subsection{Rules of Expected Values}  % 5.2.2
            \begin{enumerate}
                \item If $X$ is a random variable and $a$ and $b$ are fixed, then
                \[E(a+bX)=a+bE(X)\]
                \item If $X$ and $Y$ are random variables, then
                \[E(X+Y)=E(X)+E(Y)\]
                \item If $X$ is a random variable and $g$ is a function of $X$, then
                \[E(g(X))=\sum_{i=1}^m g(x_i)p_i\]
            \end{enumerate}

        \subsection{Variance and Standard Deviation}  % 5.2.3
            Recall sample variance measures spread by taking the average of the squared differences between observations and their center
            \begin{equation}
                s^2=\frac{\sum_{i=1}^{n}{(x_i-\bar{x})}^2}{n-1}
            \end{equation}
            \nt{Define the population variance of $X$ by $Var(X)$, $\sigma^2$, or $\sigma_X^2$.}
            \begin{equation}
                Var(X)=\sigma^2 = \sigma_X^2
            \end{equation}
            The population variance is the expected squared difference between $X$ and $\mu_X$.
            \begin{equation}
                Var(X)=E[{(X-\mu_X)}^2]=\sum{(x_i-\mu_X)}^2\cdot p_i
            \end{equation}
            Simplify.
            \begin{equation}
                Var(X)=E[{(X-\mu_X)}^2]=E(X^2)-(E(X))^2
            \end{equation}
            Then, the standard deviation is the sqaure root of the variance
            \begin{equation}
                \sigma_X=\sqrt{Var(X)}
            \end{equation}

        \subsection{Rules of Variance}  % 5.2.4


    \section{Cumulative Distribution Function}  % 5.3

    \section{Binomial Random Variable}  % 5.4
        \subsection{Binomial Experiment}  % 5.4.1
        \subsection{Binomial Probabilities}  % 5.4.2
        \subsection{Mean and Variance}  % 5.4.3

    \section{Poisson Random Variables}  % 5.5
        \subsection{Poisson Experiment}  % 5.5.1
        \subsection{Poisson Probabilities}  % 5.5.2
        \subsection{Mean and Variance}  % 5.5.3
