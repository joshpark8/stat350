\chapter{Summarizing Data Using Graphs}  % 2

\section{Variables}  % 2.1
\subsection{number of observations}  % 2.1.1
single - univariate \\
double - bivariate \\
3+ - multivariate
\subsection{type}  % 2.1.2
numerical; categorical (nominal[unordered] or ordinal[ordered]); discrete; continuous;

\section{Basics of graphing}  % 2.2
look for overall pattern; deviations; shape; center; variability

\subsection{Frequency Distribution}  % 2.2.1
\dfn{categorical variables sorted into \emph{bins} or \ital{classes}}\\
\dfn{frequency distribution consists of the \ital{frequency} of number of observations in each class}\\
\dfn{the \ital{relative frequency} is the measured frequency divided by the total data points}
\begin{equation}
    relative freq = \frac{freq}{total count}
\end{equation}

% Section 3 skipped as optional

\section{Displaying quantitative variables}  % 2.4
\subsection{Histogram - appropriate number of classes}  % 2.4.1
\[
    \# of bins \approx \sqrt{\# of observations}
\]
\subsection{Identify the shape}  % 2.4.2
peaks: unimodal, bimodal, or multimodal\\
positively skewed unimodal (peak goes left)\\
negatively skewed unimodal (peak goes right)
