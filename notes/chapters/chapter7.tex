\chapter{Sampling Distributions}  % 7

\section{Parameters and Statistics}  % 7.1
\dfn{Parameter}{a \it{\ul parameter} is a number that represents a characteristics of a \ul population}
\ex{Parameter}{$\mu, \sigma$}


\dfn{Statistic}{a \it{\ul statistic} is any number calculated from a \ul sample}
\ex{Statistic}{$\xbar, s$}
statistics are random variables because the results of each random experiment is unknown

\section{Sampling Distribution of a Sample Mean and CLT}  % 7.2
\dfn{Sampling distribution}{the probability distribution of a statistic is called a \it{sampling distribution}}
\nt{}{now have 2 distributions: sampling (for a statistic) and population (for the population)}
\subsection{What is a sampling distribution?}  % 7.2.1
sampling distr is theoretical; can never take all samples. also called probability distribution of the statistic

\subsection{The mean and standard deviation of a sampling distribution}  % 7.2.2
we use \(\xbar\) (sample mean) to make inferences about the population mean \\
\(\xbar\) varies from different samples so we must consider the distribution of \(\bar X\) \\
\nt{}{\(\bar X\) is the sample mean, so we are interested in \(\EE(\bar X)\)}

\subsection{The shape of a sampling distribution}  % 7.2.3
If a population distribution is normal, the sampling distribution is also normal
\[
    \text{If } X\sim N(\mu,\sigma\sq)\text{, then the sample distribution of }\bar X\sim N(\mu,\frac{\sigma\sq}{n})
\]
If a population is not normal, the sampling distribution is approximately normal (given large sample size)\\\\
Let $\bar X$ be the mean of observations in a random sample space of size \(n\) drawn from a population with mean \(\mu\) and finite variance \(\sigma\sq\). If the sample size \(n\) is large enough, then \(\bar X\sim N(\mu,\frac{\sigma\sq}{n})\)
