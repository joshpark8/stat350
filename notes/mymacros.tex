% General
\

\newcommand{\dfn}[1]{
    \vspace*{.5em}
    \noindent\textbf{Definition.} \hspace*{.25em}%
    \parbox[t]{\dimexpr\linewidth-7.5em\relax}{
        \hangindent=0em \hangafter=1 #1
    }
    \vspace*{.5em}
}
\newcommand{\nt}[1]{
    \vspace*{.5em}
    \noindent\textbf{Note.} \hspace*{.25em}%
    \parbox[t]{\dimexpr\linewidth-4em}{
        \hangindent=0em \hangafter=1 #1
    }
    \vspace*{.5em}
}
\newcommand{\ex}[1]{
    \vspace*{.5em}
    \noindent\textbf{Example.} \hspace*{.25em}%
    \parbox[t]{\dimexpr\linewidth-7.5em\relax}{
        \hangindent=0em \hangafter=1 #1
    }
    \vspace*{.5em}
}

% idk if or when i needed this tbh
\newcommand{\Longupdownarrow}{\Big\Updownarrow}

% For MA375 "Introduction to Discrete Mathematics"
\newcommand{\Mod}[1]{\ \mathrm{mod}\ #1}

% For MA341 "Foundations of Analysis"
\newcommand{\abs}[1]{\ensuremath{\lt\vert #1 \rt\vert}}

% For MA35301 "Linear Algebra II"
\newcommand{\set}[1]{\lt\{ #1 \rt\}}  % Set brackets

\newcommand{\sperp}{\ensuremath{S^\perp}}  % Set brackets

\newcommand{\vu}{\vec{u}}  % u vector
\newcommand{\vv}{\vec{v}}  % v vector
\newcommand{\vw}{\vec{w}}  % w vector

\newcommand{\ze}{\ul{0}}   % Zero element of vector space
\renewcommand{\u}{\ul{u}}  % u element of vector space
\renewcommand{\v}{\ul{v}}  % u element of vector space
\newcommand{\w}{\ul{w}}    % w element of vector space
\newcommand{\x}{\ul{x}}    % x element of vector space
\newcommand{\y}{\ul{y}}    % y element of vector space
\newcommand{\z}{\ul{z}}    % z element of vector space

\newcommand{\col}[2]{\begin{bmatrix} #1 \\ #2 \end{bmatrix}}  % Column vector
\newcommand{\coll}[3]{\begin{bmatrix} #1 \\ #2 \\ #3 \end{bmatrix}}  % Column vector
\newcommand{\colll}[4]{\begin{bmatrix} #1 \\ #2 \\ #3 \\ #4 \end{bmatrix}}  % Column vector

\newcommand{\llist}[3]{{#1}_{#2}, \ldots, {#1}_{#3}}  % list
\newcommand{\llistt}[3]{T({#1}_{#2}), \ldots, T({#1}_{#3})}  % List of lin transfs

\newcommand{\lincomb}[4]{{#1}_{#3}{#2}_{#3} + \cdots + {#1}_{#4}{#2}_{#4}}  % Linear combination
\newcommand{\lincombt}[3]{{#1}_1T({#2}_1) + \cdots + {#1}_{#3}T({#2}_{#3})}  % Linear combination of lin transfs

\newcommand{\inner}[2]{\ensuremath{\lt\langle #1, #2 \rt\rangle}}

\newcommand{\len}[1]{\ensuremath{\lt\vert\lt\vert #1 \rt\vert\rt\vert}}

