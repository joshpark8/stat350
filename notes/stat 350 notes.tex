\documentclass{report}
 
\usepackage[margin=1in]{geometry}
\usepackage{amsfonts}
\usepackage{amsmath}
\usepackage{amssymb}
\usepackage{amsthm}
\usepackage{CJK}
\usepackage{enumitem}
\usepackage{epsf}
\usepackage{fleqn}
\usepackage{float}
\usepackage{graphicx}
\usepackage{latexsym}
\usepackage{systeme}

% \input{mypreamble.tex}
% General
\newcommand*{\IT}[1]{\emph{#1}}
\newcommand*{\BD}[1]{\textbf{#1}}
\newcommand*{\UL}[1]{\underline{#1}}
\newcommand*{\llist}[4]{\ensuremath{#1_#2, #1_#3, \ldots, #1_#4}}

% For STAT 350 "Introduction to Statistics"
\newcommand{\dfn}[1]{
    \vspace*{.5em}
    \noindent\textbf{Definition.} \hspace*{.25em}%
    \parbox[t]{\dimexpr\linewidth-7.5em\relax}{
        \hangindent=0em \hangafter=1 #1
    }
    \vspace*{.5em}
}
\newcommand{\nt}[1]{
    \vspace*{.5em}
    \noindent\textbf{Note.} \hspace*{.25em}%
    \parbox[t]{\dimexpr\linewidth-4em}{
        \hangindent=0em \hangafter=1 #1
    }
    \vspace*{.5em}
}
\newcommand{\ex}[1]{
    \vspace*{.5em}
    \noindent\textbf{Example.} \hspace*{.25em}%
    \parbox[t]{\dimexpr\linewidth-7.5em\relax}{
        \hangindent=0em \hangafter=1 #1
    }
    \vspace*{.5em}
}
% number sets
\newcommand{\RR}[1][]{\ensuremath{\ifstrempty{#1}{\mathbb{R}}{\mathbb{R}^{#1}}}}
\newcommand{\NN}[1][]{\ensuremath{\ifstrempty{#1}{\mathbb{N}}{\mathbb{N}^{#1}}}}
\newcommand{\ZZ}[1][]{\ensuremath{\ifstrempty{#1}{\mathbb{Z}}{\mathbb{Z}^{#1}}}}
\newcommand{\QQ}[1][]{\ensuremath{\ifstrempty{#1}{\mathbb{Q}}{\mathbb{Q}^{#1}}}}
\newcommand{\CC}[1][]{\ensuremath{\ifstrempty{#1}{\mathbb{C}}{\mathbb{C}^{#1}}}}
\newcommand{\PP}[1][]{\ensuremath{\ifstrempty{#1}{\mathbb{P}}{\mathbb{P}^{#1}}}}
\newcommand{\HH}[1][]{\ensuremath{\ifstrempty{#1}{\mathbb{H}}{\mathbb{H}^{#1}}}}
\newcommand{\FF}[1][]{\ensuremath{\ifstrempty{#1}{\mathbb{F}}{\mathbb{F}^{#1}}}}
% expected value
\newcommand{\EE}{\ensuremath{\mathbb{E}}}
\newcommand{\charin}{\text{ char }}
% \DeclareMathOperator{\sign}{sign}
% \DeclareMathOperator{\Aut}{Aut}
% \DeclareMathOperator{\Inn}{Inn}
% \DeclareMathOperator{\Syl}{Syl}
% \DeclareMathOperator{\Gal}{Gal}
% \DeclareMathOperator{\GL}{GL} % General linear group
% \DeclareMathOperator{\SL}{SL} % Special linear group

%---------------------------------------
% BlackBoard Math Fonts :-
%---------------------------------------

%Captital Letters
\newcommand{\bbA}{\mathbb{A}}  \newcommand{\bbB}{\mathbb{B}}
\newcommand{\bbC}{\mathbb{C}}  \newcommand{\bbD}{\mathbb{D}}
\newcommand{\bbE}{\mathbb{E}}  \newcommand{\bbF}{\mathbb{F}}
\newcommand{\bbG}{\mathbb{G}}  \newcommand{\bbH}{\mathbb{H}}
\newcommand{\bbI}{\mathbb{I}}  \newcommand{\bbJ}{\mathbb{J}}
\newcommand{\bbK}{\mathbb{K}}  \newcommand{\bbL}{\mathbb{L}}
\newcommand{\bbM}{\mathbb{M}}  \newcommand{\bbN}{\mathbb{N}}
\newcommand{\bbO}{\mathbb{O}}  \newcommand{\bbP}{\mathbb{P}}
\newcommand{\bbQ}{\mathbb{Q}}  \newcommand{\bbR}{\mathbb{R}}
\newcommand{\bbS}{\mathbb{S}}  \newcommand{\bbT}{\mathbb{T}}
\newcommand{\bbU}{\mathbb{U}}  \newcommand{\bbV}{\mathbb{V}}
\newcommand{\bbW}{\mathbb{W}}  \newcommand{\bbX}{\mathbb{X}}
\newcommand{\bbY}{\mathbb{Y}}  \newcommand{\bbZ}{\mathbb{Z}}

%---------------------------------------
% Bold Math Fonts :-
%---------------------------------------

%Capital Letters
\newcommand{\bmA}{\boldsymbol{A}}  \newcommand{\bmB}{\boldsymbol{B}}
\newcommand{\bmC}{\boldsymbol{C}}  \newcommand{\bmD}{\boldsymbol{D}}
\newcommand{\bmE}{\boldsymbol{E}}  \newcommand{\bmF}{\boldsymbol{F}}
\newcommand{\bmG}{\boldsymbol{G}}  \newcommand{\bmH}{\boldsymbol{H}}
\newcommand{\bmI}{\boldsymbol{I}}  \newcommand{\bmJ}{\boldsymbol{J}}
\newcommand{\bmK}{\boldsymbol{K}}  \newcommand{\bmL}{\boldsymbol{L}}
\newcommand{\bmM}{\boldsymbol{M}}  \newcommand{\bmN}{\boldsymbol{N}}
\newcommand{\bmO}{\boldsymbol{O}}  \newcommand{\bmP}{\boldsymbol{P}}
\newcommand{\bmQ}{\boldsymbol{Q}}  \newcommand{\bmR}{\boldsymbol{R}}
\newcommand{\bmS}{\boldsymbol{S}}  \newcommand{\bmT}{\boldsymbol{T}}
\newcommand{\bmU}{\boldsymbol{U}}  \newcommand{\bmV}{\boldsymbol{V}}
\newcommand{\bmW}{\boldsymbol{W}}  \newcommand{\bmX}{\boldsymbol{X}}
\newcommand{\bmY}{\boldsymbol{Y}}  \newcommand{\bmZ}{\boldsymbol{Z}}
%Small Letters
\newcommand{\bma}{\boldsymbol{a}}  \newcommand{\bmb}{\boldsymbol{b}}
\newcommand{\bmc}{\boldsymbol{c}}  \newcommand{\bmd}{\boldsymbol{d}}
\newcommand{\bme}{\boldsymbol{e}}  \newcommand{\bmf}{\boldsymbol{f}}
\newcommand{\bmg}{\boldsymbol{g}}  \newcommand{\bmh}{\boldsymbol{h}}
\newcommand{\bmi}{\boldsymbol{i}}  \newcommand{\bmj}{\boldsymbol{j}}
\newcommand{\bmk}{\boldsymbol{k}}  \newcommand{\bml}{\boldsymbol{l}}
\newcommand{\bmm}{\boldsymbol{m}}  \newcommand{\bmn}{\boldsymbol{n}}
\newcommand{\bmo}{\boldsymbol{o}}  \newcommand{\bmp}{\boldsymbol{p}}
\newcommand{\bmq}{\boldsymbol{q}}  \newcommand{\bmr}{\boldsymbol{r}}
\newcommand{\bms}{\boldsymbol{s}}  \newcommand{\bmt}{\boldsymbol{t}}
\newcommand{\bmu}{\boldsymbol{u}}  \newcommand{\bmv}{\boldsymbol{v}}
\newcommand{\bmw}{\boldsymbol{w}}  \newcommand{\bmx}{\boldsymbol{x}}
\newcommand{\bmy}{\boldsymbol{y}}  \newcommand{\bmz}{\boldsymbol{z}}


\title{STAT 350 Notes}
\author{Josh Park}
\date{Summer 2024}

\begin{document}
\maketitle
\chapter{An Introduction to Statistics and Statistical Inference}

\chapter{Summarizing Data Using Graphs}

\chapter{Numerical Summary Measures}

	\section{Center of a distribution}
    \subsection{Notation}
    $x$ = random variable\\
    $x_i$ = specific observation\\
    $n$ = sample size

    \subsection{Sample mean}
    \[\bar{x}=\frac{sum\ of\ observations}{n}=\frac{1}{n}\sum x_i\]
    R command: mean(variable)

    \subsection{Sample median} 
    \[\tilde{x}= centermost\ value\ in\ ordered\ dataset\]
    R command: median(variable)

	\section{Spread or variability of the data}
	three common ways to measure spread:
	\begin{enumerate}
		\item sample range
		\item sample variance (or stdev)
		\item interquartile range (IQR)
	\end{enumerate}

    \subsection{Range}
    range = $\max(x)-\min(x)$\\
    completely depends on extreme values, so not very reliable\\
    no R command for this

    \subsection{Sample Variance (sample standard deviation)}
      \subsubsection{Variance}
      $variance=s^2_x=\frac{1}{n-1}\sum(x_i-\bar{x})^2$\\
      R command: var(variable)
      \subsubsection{Standard Deviation}
      $standard\ deviation=s_x=\sqrt{\frac{1}{n-1}\sum(x_i-\bar{x})^2}$\\
      R command: sd(variable)\\\\
      if var = sd = 0, there is no spread (all data is the same)

	\subsection{Interquartile range (IQR)}
    \subsubsection{Quartile}
    quartile = 1/4 of the data\\
    R command = quantile(variable)\\
    R command for \% = quantile(variable, prob=c (p1, p2)) 

    \subsubsection{IQR}
    IQR = $Q_3-Q_1$\\

	\section{Boxplots}
	fast way to vizualize five-number summary\\
	five number summary: minimum, first quartile, median, third quartile, maximum

    \subsection{Outliers}
    IF = inner fence\\
    OF = outer fence\\
    subscript L = lower bound\\
    subscript H = higher bound
    \begin{align}
      &IF_L=Q_1-1.5(IQR) \quad\ \ IF_H=Q_3+1.5(IQR) &\text{mild} &&\\
      &OF_L=Q_1-3(IQR) \qquad OF_H=Q_3+3(IQR) &\text{extreme}
    \end{align}

	\section{Choosing Measures of Center and Spread}
	if data is skewed, use median and IQR.\\
	if symmetric, use mean and standard deviation.

	\section{z-score}
    \subsection{z-score}
    the z-score of a data point $x_i$ quantifies distance from the mean value in terms of standard deviations.
    \[z_i=\frac{x_i-\bar{x}}{s}\] 
\end{document}
